\subsubsection*{Class: \texttt{fgo:Action}}
\label{subs:Action}
\begin{tabular}{| >{\columncolor{fast@lightgrey}}p{2.5cm}|p{12cm}|}
\hline
\textcolor{white}{\textbf{label}} & Action \\ \hline
\textcolor{white}{\textbf{description}} & An action represents a specific routine which will be performed when a certain
condition is fulfilled within a certain screen component. Examples are methods of a Web service (e.g., getItem) or functionality to update or change the contents of a form. \\ \hline
\textcolor{white}{\textbf{sub\_class\_of}} & \htmlref{\texttt{fgo:BuildingBlock}}{subs:BuildingBlock} \\ \hline
\textcolor{white}{\textbf{in\_domain\_of}} & \htmlref{\texttt{fgo:hasUse}}{subs:hasUse} \\ \hline
\textcolor{white}{\textbf{in\_range\_of}} & \htmlref{\texttt{fgo:hasAction}}{subs:hasAction} \\ \hline
\end{tabular}
\subsubsection*{Class: \texttt{fgo:BackendService}}
\label{subs:BackendService}
\begin{tabular}{| >{\columncolor{fast@lightgrey}}p{2.5cm}|p{12cm}|}
\hline
\textcolor{white}{\textbf{label}} & Backend Service \\ \hline
\textcolor{white}{\textbf{description}} & A Web service which provides data and/or functionality to a screen. The actual backend service is external to FAST, and only available through a wrapper (the service Resource). \\ \hline
\textcolor{white}{\textbf{sub\_class\_of}} & \htmlref{\texttt{fgo:BuildingBlock}}{subs:BuildingBlock} \\ \hline
\end{tabular}
\subsubsection*{Class: \texttt{fgo:BuildingBlock}}
\label{subs:BuildingBlock}
\begin{tabular}{| >{\columncolor{fast@lightgrey}}p{2.5cm}|p{12cm}|}
\hline
\textcolor{white}{\textbf{label}} & BuildingBlock \\ \hline
\textcolor{white}{\textbf{description}} & Anything that is part of a gadget. Tentatively anything that can be `touched' and moved around in the FAST IDE, from the most complex units such as screen flows, down to atomic form elements like a button or a label in a form. \\ \hline
\textcolor{white}{\textbf{super\_class\_of}} & \htmlref{\texttt{fgo:Action}}{subs:Action}, \htmlref{\texttt{fgo:BackendService}}{subs:BackendService}, \htmlref{\texttt{fgo:Condition}}{subs:Condition}, \htmlref{\texttt{fgo:Fact}}{subs:Fact}, \htmlref{\texttt{fgo:FormElement}}{subs:FormElement}, \htmlref{\texttt{fgo:Library}}{subs:Library}, \htmlref{\texttt{fgo:Pipe}}{subs:Pipe}, \htmlref{\texttt{fgo:Screen}}{subs:Screen}, \htmlref{\texttt{fgo:ScreenComponent}}{subs:ScreenComponent}, \htmlref{\texttt{fgo:ScreenFlow}}{subs:ScreenFlow}, \htmlref{\texttt{fgo:Trigger}}{subs:Trigger} \\ \hline
\textcolor{white}{\textbf{in\_domain\_of}} & \htmlref{\texttt{fgo:contains}}{subs:contains}, \htmlref{\texttt{fgo:hasCopy}}{subs:hasCopy}, \htmlref{\texttt{fgo:hasIcon}}{subs:hasIcon}, \htmlref{\texttt{fgo:hasScreenshot}}{subs:hasScreenshot}, \htmlref{\texttt{fgo:hasTemplate}}{subs:hasTemplate} \\ \hline
\textcolor{white}{\textbf{in\_range\_of}} & \htmlref{\texttt{fgo:contains}}{subs:contains}, \htmlref{\texttt{fgo:hasCopy}}{subs:hasCopy}, \htmlref{\texttt{fgo:hasTemplate}}{subs:hasTemplate} \\ \hline
\end{tabular}
\subsubsection*{Class: \texttt{fgo:Condition}}
\label{subs:Condition}
\begin{tabular}{| >{\columncolor{fast@lightgrey}}p{2.5cm}|p{12cm}|}
\hline
\textcolor{white}{\textbf{label}} & Condition \\ \hline
\textcolor{white}{\textbf{description}} & The pre- or post-condition of a certain kinds of building block. If the building block is a screen flow, each target platform will use these conditions in its own way, or may also ignore them. E.g., in EzWeb pre- and post-conditions correspond to the concepts of slot and event.
A condition can be seen as a RDF bag of facts, where every fact has to be true
for the condition be true as well. \\ \hline
\textcolor{white}{\textbf{sub\_class\_of}} & \htmlref{\texttt{fgo:BuildingBlock}}{subs:BuildingBlock} \\ \hline
\textcolor{white}{\textbf{in\_range\_of}} & \htmlref{\texttt{fgo:hasPostCondition}}{subs:hasPostCondition}, \htmlref{\texttt{fgo:hasPreCondition}}{subs:hasPreCondition} \\ \hline
\end{tabular}
\subsubsection*{Class: \texttt{fgo:Fact}}
\label{subs:Fact}
\begin{tabular}{| >{\columncolor{fast@lightgrey}}p{2.5cm}|p{12cm}|}
\hline
\textcolor{white}{\textbf{label}} & Fact \\ \hline
\textcolor{white}{\textbf{description}} & The `basic information unit of a FAST gadget'. In terms of RDF, a fact is a statement consisting of a subject, predicate and object (S, P, O). \\ \hline
\textcolor{white}{\textbf{sub\_class\_of}} & \htmlref{\texttt{fgo:BuildingBlock}}{subs:BuildingBlock} \\ \hline
\textcolor{white}{\textbf{in\_domain\_of}} & \htmlref{\texttt{fgo:hasPattern}}{subs:hasPattern}, \htmlref{\texttt{fgo:hasPatternString}}{subs:hasPatternString}, \htmlref{\texttt{fgo:isPositive}}{subs:isPositive} \\ \hline
\end{tabular}
\subsubsection*{Class: \texttt{fgo:Form}}
\label{subs:Form}
\begin{tabular}{| >{\columncolor{fast@lightgrey}}p{2.5cm}|p{12cm}|}
\hline
\textcolor{white}{\textbf{label}} & Form \\ \hline
\textcolor{white}{\textbf{description}} & A form is the visual aspect of a screen: its user interface. Each form is made up of individual form elements. \\ \hline
\textcolor{white}{\textbf{sub\_class\_of}} & \htmlref{\texttt{fgo:ScreenComponent}}{subs:ScreenComponent} \\ \hline
\textcolor{white}{\textbf{in\_domain\_of}} & \htmlref{\texttt{fgo:hasFormElement}}{subs:hasFormElement} \\ \hline
\textcolor{white}{\textbf{in\_range\_of}} & \htmlref{\texttt{fgo:hasForm}}{subs:hasForm} \\ \hline
\end{tabular}
\subsubsection*{Class: \texttt{fgo:FormElement}}
\label{subs:FormElement}
\begin{tabular}{| >{\columncolor{fast@lightgrey}}p{2.5cm}|p{12cm}|}
\hline
\textcolor{white}{\textbf{label}} & Form Element \\ \hline
\textcolor{white}{\textbf{description}} & Form elements are UI elements in a particular screen, such as buttons, lists or labels. \\ \hline
\textcolor{white}{\textbf{sub\_class\_of}} & \htmlref{\texttt{fgo:BuildingBlock}}{subs:BuildingBlock} \\ \hline
\textcolor{white}{\textbf{in\_range\_of}} & \htmlref{\texttt{fgo:hasFormElement}}{subs:hasFormElement} \\ \hline
\end{tabular}
\subsubsection*{Class: \texttt{fgo:Library}}
\label{subs:Library}
\begin{tabular}{| >{\columncolor{fast@lightgrey}}p{2.5cm}|p{12cm}|}
\hline
\textcolor{white}{\textbf{label}} & Library \\ \hline
\textcolor{white}{\textbf{description}} & Libraries are references to external code libraries required for the execution of a particular building block at runtime. \\ \hline
\textcolor{white}{\textbf{sub\_class\_of}} & \htmlref{\texttt{fgo:BuildingBlock}}{subs:BuildingBlock} \\ \hline
\textcolor{white}{\textbf{in\_domain\_of}} & \htmlref{\texttt{fgo:hasLanguage}}{subs:hasLanguage} \\ \hline
\textcolor{white}{\textbf{in\_range\_of}} & \htmlref{\texttt{fgo:hasLibrary}}{subs:hasLibrary} \\ \hline
\end{tabular}
\subsubsection*{Class: \texttt{fgo:Operator}}
\label{subs:Operator}
\begin{tabular}{| >{\columncolor{fast@lightgrey}}p{2.5cm}|p{12cm}|}
\hline
\textcolor{white}{\textbf{label}} & Operator \\ \hline
\textcolor{white}{\textbf{description}} & Operators are intended to transform and/or modify data within a screen, usually for preparing data coming from service resources for the use in the screen's interface. Operators cover different kinds of data manipulations, from simple aggregation to mediating data with incompatible schemas. \\ \hline
\textcolor{white}{\textbf{sub\_class\_of}} & \htmlref{\texttt{fgo:ScreenComponent}}{subs:ScreenComponent} \\ \hline
\textcolor{white}{\textbf{in\_range\_of}} & \htmlref{\texttt{fgo:hasOperator}}{subs:hasOperator} \\ \hline
\end{tabular}
\subsubsection*{Class: \texttt{fgo:Pipe}}
\label{subs:Pipe}
\begin{tabular}{| >{\columncolor{fast@lightgrey}}p{2.5cm}|p{12cm}|}
\hline
\textcolor{white}{\textbf{label}} & pipe or connector \\ \hline
\textcolor{white}{\textbf{description}} & Pipes are used to explicitly define the flow of data within a screen, e.g., from service resource to operator to a specific form element. \\ \hline
\textcolor{white}{\textbf{sub\_class\_of}} & \htmlref{\texttt{fgo:BuildingBlock}}{subs:BuildingBlock} \\ \hline
\textcolor{white}{\textbf{in\_domain\_of}} & \htmlref{\texttt{fgo:hasIdActionTo}}{subs:hasIdActionTo}, \htmlref{\texttt{fgo:hasIdBBFrom}}{subs:hasIdBBFrom}, \htmlref{\texttt{fgo:hasIdBBTo}}{subs:hasIdBBTo}, \htmlref{\texttt{fgo:hasIdConditionFrom}}{subs:hasIdConditionFrom}, \htmlref{\texttt{fgo:hasIdConditionTo}}{subs:hasIdConditionTo} \\ \hline
\end{tabular}
\subsubsection*{Class: \texttt{fgo:Resource}}
\label{subs:Resource}
\begin{tabular}{| >{\columncolor{fast@lightgrey}}p{2.5cm}|p{12cm}|}
\hline
\textcolor{white}{\textbf{label}} & Resource \\ \hline
\textcolor{white}{\textbf{description}} & A service resource in FAST is a wrapper around a Web service (the backend service), which makes the service available to the platform, e.g., by mapping its definition to FAST facts and actions. \\ \hline
\textcolor{white}{\textbf{sub\_class\_of}} & \htmlref{\texttt{fgo:ScreenComponent}}{subs:ScreenComponent} \\ \hline
\textcolor{white}{\textbf{in\_range\_of}} & \htmlref{\texttt{fgo:hasResource}}{subs:hasResource} \\ \hline
\end{tabular}
\subsubsection*{Class: \texttt{fgo:Screen}}
\label{subs:Screen}
\begin{tabular}{| >{\columncolor{fast@lightgrey}}p{2.5cm}|p{12cm}|}
\hline
\textcolor{white}{\textbf{label}} & Screen \\ \hline
\textcolor{white}{\textbf{description}} & An individual screen; the basic unit of user interaction in FAST. A screen is the interface through which a user gets access to data and functionality of a backend service. \\ \hline
\textcolor{white}{\textbf{sub\_class\_of}} & \htmlref{\texttt{fgo:BuildingBlock}}{subs:BuildingBlock} \\ \hline
\textcolor{white}{\textbf{in\_domain\_of}} & \htmlref{\texttt{fgo:hasForm}}{subs:hasForm}, \htmlref{\texttt{fgo:hasOperator}}{subs:hasOperator}, \htmlref{\texttt{fgo:hasResource}}{subs:hasResource} \\ \hline
\end{tabular}
\subsubsection*{Class: \texttt{fgo:ScreenComponent}}
\label{subs:ScreenComponent}
\begin{tabular}{| >{\columncolor{fast@lightgrey}}p{2.5cm}|p{12cm}|}
\hline
\textcolor{white}{\textbf{label}} & Screen Component \\ \hline
\textcolor{white}{\textbf{description}} & Screens are made up of screen components, which fundamentally include service resources, operators and forms. \\ \hline
\textcolor{white}{\textbf{super\_class\_of}} & \htmlref{\texttt{fgo:Form}}{subs:Form}, \htmlref{\texttt{fgo:Operator}}{subs:Operator}, \htmlref{\texttt{fgo:Resource}}{subs:Resource} \\ \hline
\textcolor{white}{\textbf{sub\_class\_of}} & \htmlref{\texttt{fgo:BuildingBlock}}{subs:BuildingBlock} \\ \hline
\textcolor{white}{\textbf{in\_domain\_of}} & \htmlref{\texttt{fgo:hasAction}}{subs:hasAction}, \htmlref{\texttt{fgo:hasTrigger}}{subs:hasTrigger} \\ \hline
\end{tabular}
\subsubsection*{Class: \texttt{fgo:ScreenFlow}}
\label{subs:ScreenFlow}
\begin{tabular}{| >{\columncolor{fast@lightgrey}}p{2.5cm}|p{12cm}|}
\hline
\textcolor{white}{\textbf{label}} & Screen Flow \\ \hline
\textcolor{white}{\textbf{description}} & A set of screens from which a gadget for a given target platform can be generated. \\ \hline
\textcolor{white}{\textbf{sub\_class\_of}} & \htmlref{\texttt{fgo:BuildingBlock}}{subs:BuildingBlock} \\ \hline
\end{tabular}
\subsubsection*{Class: \texttt{fgo:Trigger}}
\label{subs:Trigger}
\begin{tabular}{| >{\columncolor{fast@lightgrey}}p{2.5cm}|p{12cm}|}
\hline
\textcolor{white}{\textbf{label}} & Trigger \\ \hline
\textcolor{white}{\textbf{description}} & Triggers are the flip-side of actions. Certain events in a building block can cause a trigger to be fired. Other building blocks within the same screen, which are listening to it, will react with an action. \\ \hline
\textcolor{white}{\textbf{sub\_class\_of}} & \htmlref{\texttt{fgo:BuildingBlock}}{subs:BuildingBlock} \\ \hline
\textcolor{white}{\textbf{in\_domain\_of}} & \htmlref{\texttt{fgo:hasIdActionTo}}{subs:hasIdActionTo}, \htmlref{\texttt{fgo:hasIdBBFrom}}{subs:hasIdBBFrom}, \htmlref{\texttt{fgo:hasIdBBTo}}{subs:hasIdBBTo}, \htmlref{\texttt{fgo:hasNameFrom}}{subs:hasNameFrom} \\ \hline
\textcolor{white}{\textbf{in\_range\_of}} & \htmlref{\texttt{fgo:hasTrigger}}{subs:hasTrigger} \\ \hline
\end{tabular}
\subsubsection*{Class: \texttt{fgo:WithCode}}
\label{subs:WithCode}
\begin{tabular}{| >{\columncolor{fast@lightgrey}}p{2.5cm}|p{12cm}|}
\hline
\textcolor{white}{\textbf{label}} & With Code \\ \hline
\textcolor{white}{\textbf{description}} & Any kind of building block that can be defined as a whole through code. \\ \hline
\textcolor{white}{\textbf{in\_domain\_of}} & \htmlref{\texttt{fgo:hasCode}}{subs:hasCode}, \htmlref{\texttt{fgo:hasLibrary}}{subs:hasLibrary} \\ \hline
\textcolor{white}{\textbf{unionOf}} & \htmlref{\texttt{fgo:Form}}{subs:Form}, \htmlref{\texttt{fgo:Operator}}{subs:Operator}, \htmlref{\texttt{fgo:Resource}}{subs:Resource}, \htmlref{\texttt{fgo:Screen}}{subs:Screen} \\ \hline
\end{tabular}
\subsubsection*{Class: \texttt{fgo:WithDefinition}}
\label{subs:WithDefinition}
\begin{tabular}{| >{\columncolor{fast@lightgrey}}p{2.5cm}|p{12cm}|}
\hline
\textcolor{white}{\textbf{label}} & With Definition \\ \hline
\textcolor{white}{\textbf{description}} & Any kind of building block that can be defined declaratively in the GVS. \\ \hline
\textcolor{white}{\textbf{unionOf}} & \htmlref{\texttt{fgo:Form}}{subs:Form}, \htmlref{\texttt{fgo:Screen}}{subs:Screen} \\ \hline
\end{tabular}
\subsubsection*{Class: \texttt{fgo:WithPostConditions}}
\label{subs:WithPostConditions}
\begin{tabular}{| >{\columncolor{fast@lightgrey}}p{2.5cm}|p{12cm}|}
\hline
\textcolor{white}{\textbf{label}} & With Post-condition \\ \hline
\textcolor{white}{\textbf{description}} & Those kinds of building blocks which can have post-conditions. \\ \hline
\textcolor{white}{\textbf{in\_domain\_of}} & \htmlref{\texttt{fgo:hasPostCondition}}{subs:hasPostCondition} \\ \hline
\textcolor{white}{\textbf{unionOf}} & \htmlref{\texttt{fgo:Screen}}{subs:Screen}, \htmlref{\texttt{fgo:ScreenComponent}}{subs:ScreenComponent}, \htmlref{\texttt{fgo:ScreenFlow}}{subs:ScreenFlow} \\ \hline
\end{tabular}
\subsubsection*{Class: \texttt{fgo:WithPreConditions}}
\label{subs:WithPreConditions}
\begin{tabular}{| >{\columncolor{fast@lightgrey}}p{2.5cm}|p{12cm}|}
\hline
\textcolor{white}{\textbf{label}} & With Pre-condition \\ \hline
\textcolor{white}{\textbf{description}} & Those kinds of building blocks which can have pre-conditions. \\ \hline
\textcolor{white}{\textbf{in\_domain\_of}} & \htmlref{\texttt{fgo:hasPreCondition}}{subs:hasPreCondition} \\ \hline
\textcolor{white}{\textbf{unionOf}} & \htmlref{\texttt{fgo:Action}}{subs:Action}, \htmlref{\texttt{fgo:Screen}}{subs:Screen}, \htmlref{\texttt{fgo:ScreenFlow}}{subs:ScreenFlow} \\ \hline
\end{tabular}
