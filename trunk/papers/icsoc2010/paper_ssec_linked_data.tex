
\subsection{Serving Linked Data} % (fold)
\label{sub:linked_data}

The recently success of the web of Data has influenced the way information is now published on the web, and has been widely adopted by the academic community, large companies such as The New York Times and BBC, and national governments such as US and UK made public commitments toward open data. The main aim of the linking data is to find other related data based upon previously known data, in terms of the connections or links between them. We apply this concept in order to provide metadata about the web services as linked data, following the principles as originally defined in \cite{bizer_ijswis2009}, in order to make them available to arbitrary third-party applications. Each web service is identified by an HTTP URI and hosted in the catalogue so that it can be dereferenced through the same URI. For each building block, data is available in representations in different standard formats such as JSON (for communication with the GVS), RDF/XML, Turtle, or even HTML+RDFa as a human-readable version. The principles and best practices proposed in \cite{berrueta2008} and \cite{sauermann2008cool_uris} are also taken into account, these representations are served based on the request issued by the requesting agent, using a technique called \emph{content negotiation}. As required by the forth rule of linked data, individual building blocks link to other data on the web, thereby preventing so-called isolated ``data islands''.

To allow a third-party application to retrieve the content in the format required, we support content negotiation as stated in the \cite{http1.1}, serving the best variant for a resource, taking into account what variants are available, what variants the server may prefer to serve, what the client can accept, and with which preferences. In HTTP, this is done by the client which may send, in its request, accept headers (\texttt{Accept}, \texttt{Accept-Language} and \texttt{Accept-Encoding}), to communicate its capabilities and preferences in format, language and encoding, respectively. Concretely, the approach followed is agent-driven negotiation, where the user agent selects the specific representation for a resource.
