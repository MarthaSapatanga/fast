%!TEX root = paper.tex

\section{Introduction}
\label{sec:introduction}

Business-to-business (B2B) integration is easier said than done. Integration is a big challenge, in most cases requires a huge amount of effort, and deal with many factors and technologies for protocols, architectures, security standards, among others. Solutions based on open standards have reduced the complexity of integrating different business applications between different companies and partners. Web services and other XML standards such as RosettaNet, ebXML or OAG have been a boon to the world of B2B. By web services standards we are referring to the following open standards: Web Services Description Language (WSDL --- to describe), Universal Description, Discovery and Integration (UDDI --- to advertise and syndicate), Simple Object Access Protocol (SOAP --- to communicate) and Web Services Flow Language (WSFL --- to define work flows). Therefore, a common scenario is a company using WSDL to describe its public and private web service, publishing them either using a public or private repository using UDDI, where these web services using SOAP-based messages to achieve dynamic integration between different disparate applications.

Dealing with Business-to-consumer (B2C) integration, the adoption of these standards have offered some advantages as well, however the interaction and consumption of web services requires programming skills and understanding of many technical details about the technology, which is an obstacle for an end-user with limited knowledge on the matter (in the sense as defined in~\cite{fuchsloch2010}). 
Regarding discovery, most of the publishing and discovery platforms are syntax-based, leading to poor precision and recall in finding the correct service, making difficult to navigate through a large variety of web services~\cite{pilioura_acm2009}, 
therefore preventing end-users from performing these tasks. 
Other solutions like Semantic Web Services promised many advantages in discovery, composition and consumption of web services, independently of the provider's platform, location,  service implementation or data format. 
However, as of yet they have not been widely adopted, 
possibly because the perceived potential benefits of SWS did not justify the additional investments required for businesses. 
In a similar vein as B2C, governments are currently opening their data to the public as linked data. 
So far, the efforts are restricted to data alone, so that users left to find applications of that data. 
A natural further development in the public sector would be that their services for administration tasks and single-window applications will be opened as well. Hence, governments will be benefit from the research done around these concepts and technologies.

The motivation of our work has been strongly influenced by the end-users' needs. We are targeting users which are non-skilled in term of programming and software development, intending to empower them with tools to discover and consume web services in a straight-forward manner.
The web services and their specifications are not published right away in the platform, since this would simply mean to reimplement UDDI.
However, their definition and behaviour are wrapped in what we call a ``web service wrapper''. The resulting products of the wrapping process are two artifacts: a specific definition of the web service to be used by this tool, and a ready-to-consume (inside a web browser) piece of code with the proper functions to invoke the web service.

The remainder of this paper is organised as follows. Section 2 covers the state of the art in web services publication, discovery and wrapping/consumption. In Section 4 we present how we tackle with the web services wrapping, available possibilities and limitations. Next, section 3 describes our platform for publishing web services and empowering end-users to discover services. Section~\ref{sec:use_case} gives the reader a clearer use case and how this work is integrated into the FAST platform, and finally, we present our main conclusions and highlight future lines of research in Section 6.

