%!TEX root = paper.tex

\subsection{Simplified Model/Ontology (Knud)}
\label{ssec:ontology}
The model used to define the web service wrappers within this application has been influenced by WSDL, the main web service definition language, and some semantic-enriched approaches such as WSMO and OWL-S. There is no need to reinvent the wheel, 

A WSDL operation is mapped to an action. Its inputs and outputs are called pre-postconditions (a WSMO concept). 

This section could talk about a the part of the ontology describing the back-end services, how we remove the unnecessary information from WSDL and similar (not need anymore because the code has been generated), however we reuse some of the concepts from these technologies to define them (preconditions, postconditions, actions, and so on) allowing discovery and composition.

%\begin{figure}[ht]
%	\centering
%	\includegraphics[width=8.31cm]{uddi.png}
%	\caption{UDDI overview}
%	\label{fig:uddi}
%\end{figure}

