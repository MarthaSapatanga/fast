%!TEX root = paper.tex

\subsection{Conceptual Model}
\label{ssec:ontology}
The conceptual model used to define the web service wrappers within this application has been influenced by WSDL, the main web service definition language, and some semantic-enriched approaches such as WSMO and OWL-S. It belongs to a more complex conceptual model for FAST, which can be grouped into three levels: the gadget and screenflow level at the top, the level of individual screens in the middle, and the level of web services at the bottom. In a few words, a screenflow is a runnable set of screens, and a screen is a executable set of building blocks with a specific purpose. This section only describes in detail the former level, defining the different components a screen comprises. These three different kinds of screen components are form, operator and resource, and share the same structure. Any screen component has a set of actions, which need a set of preconditions (inputs) to be fulfilled in order to execute the action (operation in WSDL), and provides a set of postcondition (ouputs). These pre- and post-conditions are defined as individual RDF triples. This means that conditions, which are sets of facts, can be modelled as graph patterns using SPARQL notation \cite{sparql2008spec}, where a statement like ``there exists a sioc:User'' will be transformed into a simple pattern like ``a variable ?user is of type sioc:User''. 

