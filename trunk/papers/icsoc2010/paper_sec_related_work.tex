
\section{Related work}
\label{sec:related_work}

Web services have been around for a long time since they first appeared. One of their most important claims has been (syntactic) interoperability between third-party systems and applications based on different platforms / programming languages. System integration, within a company or between systems from different enterprises became easier with the adoption of the web services standard technologies, being the Web Service Description or WSDL the most extended standard used for the definition of web services. Many integrated development environments (IDEs) can deal with WSDL to facilitate the integration task, however it is still required that developers read and understand their specifications, and implement part of the logic in a programmatically-fashion way. A step forward, there are several tools which facilitate the interaction with data sources and services on the Web. Yahoo! Pipes, Apatar, JackBe Presto, Microsoft Popfly and NetVibes, among others provide a set of modules to access different kind of data sources, such as RSS feeds, a given web page (HTML code), Flickr images, Google base or the Yahoo! search engine, databases (MySQL, PostgreSql, Oracle), and powerful enterprise systems such as Salesforce CRM, SugarCRM and Goldmine CRM. However, none of the solutions in the market really facilitate end-users to build up their own applications or widgets allowing the interaction with Web services created by third-party providers. The solutions found are mainly data-oriented (RSS feeds, databases, raw text). Several tools permits some sort of (Web) service integration, but are meant to be used within an entreprise level by savvy business users or developers. [TODO: rewrite this paragraph] This paper presents an approach to leverage the possibility of integrate REST or SOAP-based web service inside a widget or any application for a web browser (by creating service wrappers), and graphical tool will be provided to facilitate the process of building these service wrappers.

In the context of publishing and discovering Web services, a service provider have well-known and widely used technologies to accomplish the task of publication, such as the Universal Description, Discovery and Integration (UDDI) or WS-Discovery. The UDDI service registry specification \cite{uddi2004} is currently one of the core standards in the Web service technology stack and an integral part of every major SOA vendor's technology strategy and offers to requesters the ability to discover services via the Internet. In a few words, UDDI serves as a centralised repository of WSDL documents. A similar concept is iServe \cite{pedrinaci_ores2010}. This platform aims to publish Web services as what they called Lined Services Linked Services --linked data describing services, storing Web service definitions in a semantic-annotation fashion so semantic Web technologies for Web service discovery and processing can make use of them. However, the platform does not deal with the step from the definition to the consumption of the services.

A different approach for Web service discovery is the Web Services Dynamic Discovery (WS-Discovery) specification. The core of this approach is a multicast discovery protocol. Service providers and consumers  listen each other for new services specifications within a network, so there is no need of a centralised registry. As a drawback, WS-Discovery do not support Internet-scale discovery, therefore make if useless for our purposes.
