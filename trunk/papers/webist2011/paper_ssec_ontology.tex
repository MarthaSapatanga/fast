%!TEX root = paper.tex

\subsection{Conceptual Model}
\label{ssec:ontology}
The conceptual model used to define the web service wrappers within this application has been influenced by both WSDL and semantic approaches such as WSMO and OWL-S. 
It is part of a more complex conceptual model for FAST~\cite{moeller2010fast_ontology}, which can be grouped into three levels: the gadget and screenflow level at the top, the level of individual screens in the middle, and the level of web services at the bottom. 
All these building blocks and their sub-parts
% --- \emph{forms} for the UI, \emph{operators} for data manipulation and \emph{back-end services} --- 
share the same structure: a set of actions (like operations in WSDL), each of which needs a set of pre-conditions (inputs) to be fulfilled in order to be executed, and provides a set of post-conditions (outputs) to other building blocks. These pre- and post-conditions are defined as RDF graph patterns. E.g., the post-condition of a login service such as ``there exists a user'' will be expressed as a simple pattern such as ``\texttt{?user a sioc:User}''\footnote{We use Trig (\url{http://www4.wiwiss.fu-berlin.de/bizer/TriG/}) and SPARQL (\url{http://www.w3.org/TR/rdf-sparql-query/}) notation for RDF graphs throughout this paper.}. Using this mechanism, extended by RDFS entailment rules, services and other building blocks can be matched automatically. The publishing and discovery platform can employ this functionality to support the internal discovery of web services based on the current user needs (expressed in the same way as pre-conditions).

