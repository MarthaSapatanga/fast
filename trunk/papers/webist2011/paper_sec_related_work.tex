%!TEX root = paper.tex

\section{Related work}
\label{sec:related_work}

Web services have been around for a long time since they first appeared.
One of the most important claims about their benefits has been (syntactic) interoperability between third-party systems and applications based on different platforms / programming languages. 
System integration, within a company or between systems from different enterprises, became easier with the adoption of web service standard technologies, in particular the Web Service Description Language (WSDL), the most extensive standard used for the definition of web services. 
Many integrated development environments (IDEs) can deal with WSDL to facilitate the integration task, however it is still required that developers read and understand their specifications, and implement part of the logic programmatically. 
As a step forward, there are several tools which facilitate the interaction with data sources and services on the Web. Yahoo! Pipes, Apatar, JackBe Presto, Microsoft Popfly (now defunct) and NetVibes, among others provide a set of modules to access different kind of data sources, such as RSS feeds, a given web page (HTML code), Flickr images, Google base or the Yahoo! search engine, databases (MySQL, PostgreSql, Oracle), and powerful enterprise systems such as Salesforce CRM, SugarCRM and Goldmine CRM. 
However, none of the solutions in the market really facilitates end-users to build  their own applications or widgets allowing the interaction with web services created by third-party providers. The solutions found are mainly data-oriented (RSS feeds, databases, raw text). Several tools permits some sort of (web) service integration, but are meant to be used within an enterprise level by savvy business users or developers. The solution presented in this paper leverage the possibility of integrating REST or SOAP-based web service inside visual applications running in a browser (e.g. widgets) by providing a platform to create, publish and discover web services wrappers.

In the context of publishing and discovering Web services, service providers have well-known and widely used technologies to accomplish the task of publication, such as the Universal Description, Discovery and Integration (UDDI) or WS-Discovery. The UDDI service registry specification~\cite{uddi2004} is currently one of the core standards in the Web service technology stack and an integral part of every major SOA vendor's technology strategy and offers to requesters the ability to discover services via the Internet. 
In short, UDDI serves as a centralised repository of WSDL documents. A similar concept is iServe~\cite{pedrinaci_ores2010}. This platform aims to publish web services as what they called Linked Services --- linked data describing services ---, storing web service definitions as semantic annotations, so that other semantic web-aware applications may take advantage of it. However, the platform does not deal with the step from the definition to the consumption of the services.

A different approach for web service discovery is the Web Services Dynamic Discovery (WS-Discovery) specification~\cite{beatty2005}. The core of this approach is a multicast discovery protocol. Service providers and consumers listen to each other for new services specifications within a network, so there is no need of a centralised registry. As a drawback, WS-Discovery do not support Internet-scale discovery, making it useless for our purposes.
