%!TEX root = paper.tex

\section{Publishing and discovery platform}
\label{sec:discovery}

The area of web service publication and discovery has been subject for a lot of research since the very beginning the concept was coined. The current state of the art provides several solutions and strategies which providers and consumers may take advantage of. However, as explained in previous sections, a large number of them suffer from limited syntax-based descriptions and simple keyword-based search, and the gap between discovery and consumption is an obstacle for their usage by non-technical end-users. In order to improve this issue, this paper presents a novel publishing platform, permitting any enterprise or individual to publish their public domain web services, with an enhanced semantic search based on the definition of the services, and serving web service wrappers for easy consumption in web applications.

\subsection{Overview}
\label{ssec:overview}

The main difference of this platform with regards to the state of the art is that it is targeting a different kind of user, and covers many deficiencies others tools do not tackle. As a brief overview, the platform being presented:

\begin{itemize}
  \item allows functional discovery through web service pre- and post-conditions;
	\item serves web services wrappers ready to consume in web applications;
	\item provides advanced search capabilities, based on the formal service definition and inferences extracted from it;
	\item supports managing its resources via its RESTful API;
	\item offers a SPARQL endpoint, giving direct access to the data through complex queries;
	\item offers web service descriptions as linked data;
	\item follows well-known best practices for publishing data on the web;
	\item supports content negotiation so that different clients may retrieve information in their preferred format, choosing from JSON, RDF/XML, RDF/N3 or a human-readable HTML version.
\end{itemize}

This platform was intended and it is being used as part of the storyboard tool of the FAST platform~\cite{hoyer2009fast}, and it communicates with other components through its RESTful API, using JSON for the input of the requests. This paper will focus on the functionality regarding web service publication and discovery, and the process to publish any third-party web service into the platform. This process requires two steps:
\begin{inparaenum}[(i)]
	\item generation of the wrapper (see Sect. \ref{sec:wrapping_web_services}), and 
	\item creation of the resource in the publishing platform. 
\end{inparaenum}
The following is an example of part of a request sent to create a new wrapper into the platform (for the sake of clarity, some parts of the request have been omitted, but the main idea is shown):

\begin{listing}
\begin{verbatim}
{
  "code": "http://demo.fast.morfeo-project.org/code/amazonSearch.js",
  "creationDate": "2010-01-26T17:01:13+0000",
  "description": {"en-gb": "Amazon web service - Search"},
  "label": {"en-gb": "Amazon search"},
  "actions": [{
    "name": "filter",
    "preconditions": [[{
      "id": "item",
      "label": {"en-gb": "Ebay List"},
      "pattern": "?item
                  http://www.w3.org/1999/02/22-rdf-syntax-ns#type
                  http://aws.amazon.com/AWSECommerceService#Item",
      "positive": true
  }]],
  "postconditions": [
    ...
  ],
  ...
\end{verbatim}
\label{lis:json_request}
\end{listing}

The enriched search capabilities are supported by the definition of pre- and post-conditions. They allows to define the inputs and outputs of the services and other building blocks using concepts from any ontology, and in this way to find web services or other building blocks which match each other and can therefore be integrated. The concepts of pre-/post-conditions were strongly influenced by WSMO~\cite{roman2005}, simplified and implemented in RDFS for better live performance. The model used to define the web service wrappers and the ontology created are described in more detail in Sect.~\ref{ssec:ontology}.

\subsection{Platform architecture}
\label{ssec:architecture}

The architecture comprises an RDF store used for persistence, a business layer dealing with the model and reasoning, and a public facade providing the core functionality as a RESTful API, as well as a SPARQL endpoint accessing the RDF store directly. This presentation layer is aimed to interact with the FAST Gadget Visual Storyboard (see Sect.~\ref{sec:use_case}), or any other third-party application.

The RESTful API and the SPARQL endpoint are part of the presentation layer. The SPARQL endpoint is offered using the SPARQL protocol service as defined in the SPROT specification~\cite{sprot} and is aimed to enable third-party developers to directly query the knowledge base using SPARQL queries. This feature is supported by the Sesame RESTful HTTP interface for SPARQL Protocol for RDF.

The business logic layer contains all the domain-specific processing and reasoning. It provides functions to interact with all the elements of the domain model specified in \ref{ssec:ontology},
acting as a mediator between the public RESTful API and the persistence layer.

The persistence layer provides an API allowing to work with a standard set of objects representing the model. The interaction with the underlying RDF store is made via RDF2Go, an abstraction over triple (and quad) stores, which allows to program against RDF2Go interfaces and choose any RDF store implementation. This allows having a completely extensible and configurable framework for storage mechanisms, inferencers, RDF file formats, query result formats and query languages.


\subsection{Simplified Model/Ontology (Knud)}
The model used to define the web service wrappers within this application has been influenced by WSDL, the main web service definition language, and some semantic-enriched approaches such as WSMO and OWL-S. There is no need to reinvent the wheel, 

A WSDL operation is mapped to an action. Its inputs and outputs are called pre-postconditions (a WSMO concept). 

This section could talk about a the part of the ontology describing the back-end services, how we remove the unnecessary information from WSDL and similar (not need anymore because the code has been generated), however we reuse some of the concepts from these technologies to define them (preconditions, postconditions, actions, and so on) allowing discovery and composition.

%\begin{figure}[ht]
%	\centering
%	\includegraphics[width=8.31cm]{uddi.png}
%	\caption{UDDI overview}
%	\label{fig:uddi}
%\end{figure}




\subsection{Serving Linked Data} % (fold)
\label{sub:linked_data}

The recently success of the web of Data has influenced the way information is now published on the web, and has been widely adopted by the academic community, large companies such as The New York Times and BBC, and national governments such as US and UK made public commitments toward open data. The main aim of the linking data is to find other related data based upon previously known data, in terms of the connections or links between them. We apply this concept in order to provide metadata about the web services as linked data, following the principles as originally defined in \cite{bizer_ijswis2009}, in order to make them available to arbitrary third-party applications. Each web service is identified by an HTTP URI and hosted in the catalogue so that it can be dereferenced through the same URI. For each building block, data is available in representations in different standard formats such as JSON (for communication with the GVS), RDF/XML, Turtle, or even HTML+RDFa as a human-readable version. The principles and best practices proposed in \cite{berrueta2008} and \cite{sauermann2008cool_uris} are also taken into account, these representations are served based on the request issued by the requesting agent, using a technique called \emph{content negotiation}. As required by the forth rule of linked data, individual building blocks link to other data on the web, thereby preventing so-called isolated ``data islands''.

To allow a third-party application to retrieve the content in the format required, we support content negotiation as stated in the \cite{http1.1}, serving the best variant for a resource, taking into account what variants are available, what variants the server may prefer to serve, what the client can accept, and with which preferences. In HTTP, this is done by the client which may send, in its request, accept headers (\texttt{Accept}, \texttt{Accept-Language} and \texttt{Accept-Encoding}), to communicate its capabilities and preferences in format, language and encoding, respectively. Concretely, the approach followed is agent-driven negotiation, where the user agent selects the specific representation for a resource.

