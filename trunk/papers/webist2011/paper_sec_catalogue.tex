%!TEX root = paper.tex

\section{Publishing and discovery platform}
\label{sec:discovery}

The area of web service publication and discovery has been subject for a lot of research since the very beginning the concept was coined. The current state of the art provides several solutions and strategies which providers and consumers may take advantage of. However, as explained in previous sections, a large number of them suffer from limited syntax-based descriptions and simple keyword-based search, and the gap between discovery and consumption is an obstacle for their usage by non-technical end-users. In order to improve this issue, this paper presents a novel publishing platform, permitting any enterprise or individual to publish their public domain web services, with an enhanced semantic search based on the definition of the services, and serving web service wrappers for easy consumption in web applications.

% Overview subsection
\subsection{Overview}
\label{ssec:overview}

The main difference of this platform with regards to the state of the art is that it is targeting a different kind of user, and covers many deficiencies others tools do not tackle. As a brief overview, the platform being presented:

\begin{itemize}
  \item allows functional discovery through web service pre- and post-conditions;
	\item serves web services wrappers ready to consume in web applications;
	\item provides advanced search capabilities, based on the formal service definition and inferences extracted from it;
	\item supports managing its resources via its RESTful API;
	\item offers a SPARQL endpoint, giving direct access to the data through complex queries;
	\item offers web service descriptions as linked data;
	\item follows well-known best practices for publishing data on the web;
	\item supports content negotiation so that different clients may retrieve information in their preferred format, choosing from JSON, RDF/XML, RDF/N3 or a human-readable HTML version.
\end{itemize}

This platform was intended and it is being used as part of the storyboard tool of the FAST platform~\cite{hoyer2009fast}, and it communicates with other components through its RESTful API, using JSON for the input of the requests. This paper will focus on the functionality regarding web service publication and discovery, and the process to publish any third-party web service into the platform. This process requires two steps:
\begin{inparaenum}[(i)]
	\item generation of the wrapper (see Sect. \ref{sec:wrapping_web_services}), and 
	\item creation of the resource in the publishing platform. 
\end{inparaenum}
The following is an example of part of a request sent to create a new wrapper into the platform (for the sake of clarity, some parts of the request have been omitted, but the main idea is shown):

\begin{listing}
\begin{verbatim}
{
  "code": "http://demo.fast.morfeo-project.org/code/amazonSearch.js",
  "creationDate": "2010-01-26T17:01:13+0000",
  "description": {"en-gb": "Amazon web service - Search"},
  "label": {"en-gb": "Amazon search"},
  "actions": [{
    "name": "filter",
    "preconditions": [[{
      "id": "item",
      "label": {"en-gb": "Ebay List"},
      "pattern": "?item
                  http://www.w3.org/1999/02/22-rdf-syntax-ns#type
                  http://aws.amazon.com/AWSECommerceService#Item",
      "positive": true
  }]],
  "postconditions": [
    ...
  ],
  ...
\end{verbatim}
\label{lis:json_request}
\end{listing}

The enriched search capabilities are supported by the definition of pre- and post-conditions. They allows to define the inputs and outputs of the services and other building blocks using concepts from any ontology, and in this way to find web services or other building blocks which match each other and can therefore be integrated. The concepts of pre-/post-conditions were strongly influenced by WSMO~\cite{roman2005}, simplified and implemented in RDFS for better live performance. The model used to define the web service wrappers and the ontology created are described in more detail in Sect.~\ref{ssec:ontology}.

% Architecture subsection
\subsection{Platform architecture}
\label{ssec:architecture}

The architecture comprises an RDF store used for persistence, a business layer dealing with the model and reasoning, and a public facade providing the core functionality as a RESTful API, as well as a SPARQL endpoint accessing the RDF store directly. This presentation layer is aimed to interact with the FAST Gadget Visual Storyboard (see Sect.~\ref{sec:use_case}), or any other third-party application.

The RESTful API and the SPARQL endpoint are part of the presentation layer. The SPARQL endpoint is offered using the SPARQL protocol service as defined in the SPROT specification~\cite{sprot} and is aimed to enable third-party developers to directly query the knowledge base using SPARQL queries. This feature is supported by the Sesame RESTful HTTP interface for SPARQL Protocol for RDF.

The business logic layer contains all the domain-specific processing and reasoning. It provides functions to interact with all the elements of the domain model specified in \ref{ssec:ontology},
acting as a mediator between the public RESTful API and the persistence layer.

The persistence layer provides an API allowing to work with a standard set of objects representing the model. The interaction with the underlying RDF store is made via RDF2Go, an abstraction over triple (and quad) stores, which allows to program against RDF2Go interfaces and choose any RDF store implementation. This allows having a completely extensible and configurable framework for storage mechanisms, inferencers, RDF file formats, query result formats and query languages.


% Conceptual model subsection

\subsection{Simplified Model/Ontology (Knud)}
The model used to define the web service wrappers within this application has been influenced by WSDL, the main web service definition language, and some semantic-enriched approaches such as WSMO and OWL-S. There is no need to reinvent the wheel, 

A WSDL operation is mapped to an action. Its inputs and outputs are called pre-postconditions (a WSMO concept). 

This section could talk about a the part of the ontology describing the back-end services, how we remove the unnecessary information from WSDL and similar (not need anymore because the code has been generated), however we reuse some of the concepts from these technologies to define them (preconditions, postconditions, actions, and so on) allowing discovery and composition.

%\begin{figure}[ht]
%	\centering
%	\includegraphics[width=8.31cm]{uddi.png}
%	\caption{UDDI overview}
%	\label{fig:uddi}
%\end{figure}




% Discovery and recommendations subsection
%!TEX root = paper.tex

\subsection{Discovery mechanisms} % (fold)
\label{ssec:discovery}

The platform presented in this paper is intended to support web service discovery and recommendation capabilities.
As explained in detail in Section~\ref{ssec:ontology}, there are different types of building block which can be modelled
using the platform, leading to a necessity for different kind of discovery or recommendation mechanisms, depending
on the level of composition to deal with: screen-flow or screen. In the screen-flow composition, the smallest
functional unit is the screen, hance a screen-flow is composed by interconnecting a number of screens between 
each other. In fact, this composition can be applied to any sort of building block since they all share the
same structure: a set of pre- and post-conditions which make possible the communication with other building blocks.
That said, technically the algorithms presented in this section could be used for the discovery and composition of
any type of building block stored in the platform.

At the screen-flow design level, the composition is made using a set of screens, and the pre- and post-conditions 
of the screen-flow itself, used as entry and exit points. A screen is only reachable when its pre-conditions are satisfied
as well, meaning that the post-conditions of another screen, or the pre-conditions of the screen-flow are compatible 
with them.

The main goal of the discovery process is to help the user to make the screen-flow executable, in other words, to make
all the screens within a screen-flow reachable. To do that, the platform offers two mechanisms to find and recommend
screens, previously stored in the building block base (the catalogue), which will make the screen-flow executable: a
simple discovery based on pre- and postconditions, and a multi-step discovery or planning mechanism. Like any other
search engine or recommender system, the results obtained in the discovery process need to rank before being presented
to the user. The ranking algorithm is covered in Section~\ref{sssec:ranking}. 

\subsubsection{Simple discovery based on pre- and post-conditions}
\label{sssec:simple_discovery}

At a particular state of a composition, ie. at screen-flow design, several of the building blocks used to compose it, 
screens in the case of a screen-flow, might be unreachable, therefore there must be pre-conditions not satisfied. The
platform will assist the process by recommending building blocks which will satisfy these pre-conditions. In a nutshell,
it collects the pre-conditions of all the unreachable building blocks, used as a graph pattern (see Section~\ref{ssec:ontology})
in order to be matched against the post-conditions of any building block. In the following scenario, there are two 
screens: \emph{s1} and \emph{s2}. \emph{s1} has as a pre-condition: ``there exists a search criteria'', and as a 
post-condition: ``there exists a item''. \emph{s2} has just a pre-condition stating: ``there exists a search criteria''.
The same example formally defined:

\begin{listing}
\begin{verbatim}
:G1 { :s1 a fgo:Screen .      
      :s1 fgo:hasPrecondition c1 .
      :s1 fgo:hasPostcondition c2 .
      :c1 fgo:hasPattern GC1 .
      :c2 fgo:hasPattern GC2 .
      :s2 a fgo:Screen .      
      :s2 fgo:hasPrecondition c3 .
      :c3 fgo:hasPattern GC3 }
:GC1 { _:x a amazon:SearchCriteria }
:GC2 { _:x a amazon:Item }
:GC3 { _:x a amazon:SearchCriteria }
\end{verbatim}
\label{lis:discovery_example}
\end{listing}

The algorithm will construct a SPARQL query to retrieve building blocks satisfying the pre-conditions \emph{c1} and \emph{c3}. 
It would look something like:

\begin{listing}
\begin{verbatim}
PREFIX rdf: <http://www.w3.org/1999/02/22-rdf-syntax-ns#>
PREFIX fgo: <http://purl.oclc.org/fast/ontology/gadget#>
PREFIX amazon:<http://aws.amazon.com/AWSECommerceService#>

SELECT DISTINCT ?bb 
WHERE { 
  ?bb rdf:type fgo:Screen . 
  {
    {
      ?bb fgo:hasPostCondition ?c .
      ?c fgo:hasPattern ?p .
      GRAPH ?p { ?x rdf:type amazon:SearchCriteria } 
    }
    UNION
    {
      ?bb fgo:hasPostCondition ?c .
      ?c fgo:hasPattern ?p .
      GRAPH ?p { ?x rdf:type amazon:Item } 
    }
  }
  FILTER (?bb != <http://fast.org/screens/S1>) 
  FILTER (?bb != <http://fast.org/screens/S2>) 
}
\end{verbatim}
\label{lis:sparql_find}
\end{listing}

However, you might have realised that the pre-condition \emph{c3} would be satisfied whether the screen \emph{s1} were 
reachable, hence it will removed those pre-conditions that would be satisfied by the current elements being used in
the composition. The removal of unnecessary pre-conditions improves the performance of the algorithm, and it leads to find
better results for the user, since the focus is given to make satisfy the pre-conditions which could not be satisfied
by any of the elements of the current composition.

\subsubsection{Enhanced discovery: search tree planning}
\label{sssec:planning}

In artificial intelligence, the term \emph{planning} originally meant a search for a sequence of logical operators or actions
that transform an initial world state into a desired goal state. Presently, it also includes many decision-theoretic ideas,
imperfect state information, and game-theoretic equilibria.

This paper applies the concept of planning to the design of building blocks. Back to the screen-flow design, the goal is to
make the screen-flow executable, the initial state is a certain screen which is not yet reachable, and the plans are sets of 
screens which are reachable by themselves, and accomplish the goal. The algorithm has been influenced by the ideas behing the 
heuristic search. For a certain state, i.e. the initial state which contains the screen to make reachable, a large tree of 
possible continuations is considered, in fact any screen would fit. Those screens which post-conditions do not satisfy in any
form the unsatisfied pre-conditions are discarded, reducing the branches of the tree. A branch stops growing when a screens is
already reachable (i.e. it has no pre-conditions) becoming a leaf of the tree. Once there are not screens added in a certain
step, the algorithm stops and discards the incomplete branches, in other words, those branches where the leaf is not reachable.

It is worth pointing out some of the tree structure is pre-computed beforehand. Any time a building block is inserted into the
catalogue, the algorithm is executed following two approaches: \emph{forward search} and \emph{backward search}. The forward
search approach finds the building blocks whose pre-conditions will be satisfied by the post-conditions of the new building 
block while the backward search finds the building blocks whose post-conditions will satisfy the pre-conditions of the recently
created building block. This data is stored and used for the tree or plan builder algorithm, without needing to chec the
compatibility of pre- and post-conditions for every single building block at every request, increasing highly the performance. 

\subsubsection{Results ranking}
\label{sssec:ranking}

In the previous sections there have been presented two ways of building block discovery for the composition or design of new
building blocks. The ensure compatibility based on the functional specification of the building blocks, however both are leaving
out the measurement of quality of the results. In other words, it can be seen as a search engine, returning a set of results
for a certain input, and as any other single search engine, a ranking mechanism is needed.

The simple discovery algorithm, in order to rank the results, applies the following rules:
\begin{enumerate}
 \item it gives a higher position to those building blocks which satisfy the highest number of pre-conditions;
 \item it prioritise building blocks created by the same user who is querying;
 \item it adjust the rank by using the ratings given to the building blocks, and their popularity in terms of usage statistics;
 \item it weights the results according to non-functional features such as availability. This is only applied for what we call
``web service wrapper'', and it is calculated periodically by invoking the wrapped web services generating an up-time rate.
\end{enumerate}

For the planning case, the objective is not only to produce a plan but also to satisfy user-specified preferences, or what is
known as \emph{preference-based planning}. The ranking algorithm:
\begin{enumerate}
 \item minimizes the size of the plans, after removing the elements of the plan which are already in the canvas, so it gives
priority to the elements the user has already inserted,
 \item adjust the rank by using the rules 2, 3 and 4 used in the ranking algorithm of simple discovery.
\end{enumerate}



% Conceptual model subsection

\subsection{Serving Linked Data} % (fold)
\label{sub:linked_data}

The recently success of the web of Data has influenced the way information is now published on the web, and has been widely adopted by the academic community, large companies such as The New York Times and BBC, and national governments such as US and UK made public commitments toward open data. The main aim of the linking data is to find other related data based upon previously known data, in terms of the connections or links between them. We apply this concept in order to provide metadata about the web services as linked data, following the principles as originally defined in \cite{bizer_ijswis2009}, in order to make them available to arbitrary third-party applications. Each web service is identified by an HTTP URI and hosted in the catalogue so that it can be dereferenced through the same URI. For each building block, data is available in representations in different standard formats such as JSON (for communication with the GVS), RDF/XML, Turtle, or even HTML+RDFa as a human-readable version. The principles and best practices proposed in \cite{berrueta2008} and \cite{sauermann2008cool_uris} are also taken into account, these representations are served based on the request issued by the requesting agent, using a technique called \emph{content negotiation}. As required by the forth rule of linked data, individual building blocks link to other data on the web, thereby preventing so-called isolated ``data islands''.

To allow a third-party application to retrieve the content in the format required, we support content negotiation as stated in the \cite{http1.1}, serving the best variant for a resource, taking into account what variants are available, what variants the server may prefer to serve, what the client can accept, and with which preferences. In HTTP, this is done by the client which may send, in its request, accept headers (\texttt{Accept}, \texttt{Accept-Language} and \texttt{Accept-Encoding}), to communicate its capabilities and preferences in format, language and encoding, respectively. Concretely, the approach followed is agent-driven negotiation, where the user agent selects the specific representation for a resource.

