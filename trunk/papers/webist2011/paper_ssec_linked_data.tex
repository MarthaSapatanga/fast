%!TEX root = paper.tex

\subsection{Serving Linked Data} % (fold)
\label{sub:linked_data}

The idea of a Web of data has recently seen a remarkable uptake, e.g. highlighted by large players such as the New York Times, the BBC or an increasing number of national governments (most notably the US and UK governments).
We apply this concept in order to provide metadata about the web services as linked data, following the principles as defined in~\cite{bizer_ijswis2009}, in order to make them available to arbitrary third-party applications. Each web service is identified by an HTTP URI and hosted in the publishing platform so that it can be dereferenced through the same URI. For each building block, data is available in representations in different standard formats such as JSON (for communication with the applications such as the widget designer shown in Fig.~\ref{fig:fast_architecture}), RDF/XML, Turtle, or even HTML+RDFa as a human-readable version. To do this, we employ content-negotiation, such as proposed as a best practice in \cite{sauermann2008cool_uris}.
%The principles and best practices proposed in \cite{berrueta2008} and  are also taken into account, so that these representations are served based on the request issued by the requesting agent, using a technique called \emph{content negotiation}. 
%As suggested by linked data principles, individual building blocks link to other data on the web, thereby preventing so-called isolated ``data islands''.

% To allow a third-party application to retrieve the content in the format required, we support content negotiation as defined in \cite{http1.1} and proposed as best practice in \cite{sauermann2008cool_uris}, serving the best variant for a resource, taking into account what variants are available, what variants the server may prefer to serve, what the client can accept, and with which preferences.
% In HTTP, this is done by the client which may send, in its request, accept headers (\texttt{Accept}, \texttt{Accept-Language} and \texttt{Accept-Encoding}), to communicate its capabilities and preferences in format, language and encoding, respectively. Concretely, the approach followed is agent-driven negotiation, where the user agent selects the specific representation for a resource.
