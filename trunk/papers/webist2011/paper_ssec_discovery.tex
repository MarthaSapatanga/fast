%!TEX root = paper.tex

\subsection{Discovery Mechanisms} % (fold)
\label{ssec:discovery}

% Due to the diversity of building blocks available, there is a necessity for different kinds of discovery or recommendation mechanisms, depending on the level of composition: screen-flow or screen.

The main goal of the discovery process is to aid the user in finding suitable building blocks to complement the ones they are already using. E.g., on the screen-flow level, this would mean to suggest additional screens to make existing screens within a screen-flow reachable, and the screen-flow therefore executable.
The platform offers two mechanisms to find and recommend screens (or other building blocks) stored in the catalogue: a simple discovery based on pre- and postconditions, and a multi-step discovery or planning algorithm. 
Before being presented to the user, the results are being ranked, as discussed in Sect.~\ref{sssec:ranking}. 

\subsubsection{Simple Discovery Based on Pre- and Post-Conditions}
\label{sssec:simple_discovery}

In this simple approach, the platform will assist the process by recommending all building blocks which will satisfy currently unfulfilled pre-conditions. 
The pre-conditions of all the unreachable building blocks are collected as a graph pattern, which is then matched against the post-conditions of all available building blocks. 
In the following scenario, there are two 
screens: \emph{s1} and \emph{s2}. \emph{s1} has as a pre-condition: \emph{``there exists a search criteria''}, and as a 
post-condition: \emph{``there exists a item''}. \emph{s2} has just a pre-condition stating: \emph{``there exists a search criteria''}.

\lstset{
	language=sparql
}
\begin{lstlisting}
:G1 { :s1 a fgo:Screen .      
      :s1 fgo:hasPrecondition c1 .
      :s1 fgo:hasPostcondition c2 .
      :c1 fgo:hasPattern GC1 .
      :c2 fgo:hasPattern GC2 .
      :s2 a fgo:Screen .      
      :s2 fgo:hasPrecondition c3 .
      :c3 fgo:hasPattern GC3 }
:GC1 { _:x a amazon:SearchCriteria }
:GC2 { _:x a amazon:Item }
:GC3 { _:x a amazon:SearchCriteria }
\end{lstlisting}

The algorithm will construct a SPARQL query to retrieve building blocks satisfying the pre-conditions \emph{c1} and \emph{c3}. 
The query, although simplified for the sake of clarity, would look something like:

\begin{lstlisting}
SELECT DISTINCT ?bb 
WHERE { 
  ?bb a fgo:Screen . 
  { { ?bb fgo:hasPostCondition ?c .
      ?c fgo:hasPattern ?p .
      GRAPH ?p { ?x a amazon:SearchCriteria } 
    } UNION {
      ?bb fgo:hasPostCondition ?c .
      ?c fgo:hasPattern ?p .
      GRAPH ?p { ?x a amazon:Item } } }
  FILTER (?bb != <http://fast.org/screens/S1>) 
  FILTER (?bb != <http://fast.org/screens/S2>) }
\end{lstlisting}

% However, in the example presented, it is straight-forward to spot that the pre-condition \emph{c3} would be satisfied whether the screen \emph{s1} was reachable, being unnecessary to query the recommender algorithm to retrieve screens satisfying this pre-condition, hence these pre-conditions are removed for the construction of the query reducing the complexity of the problem, hence improving the overall performance of the algorithm. Moreover, it leads to find better results for the user, since the focus is given to make satisfy the pre-conditions which could not be satisfied by any of the elements of the current composition.

\subsubsection{Enhanced Discovery: Search Tree Planning}
\label{sssec:planning}

In artificial intelligence, the term \emph{planning} originally meant to search for a sequence of logical operators or actions that transform an initial state into a desired goal state.% Presently, it also includes many decision-theoretic ideas, imperfect state information, and game-theoretic equilibria.

In contrast to the simple approach, the planning approach finds sets of building blocks to fulfil the pre-conditions of a given building block (e.g., a screen). For a certain state, i.e., the initial 
state which contains the pre-condition to fulfil, a large search tree of possible continuations is considered. Those building blocks cannot satisfy the unsatisfied pre-conditions are discarded, reducing the branches of the tree. A branch stops growing when a building block is reachable (i.e., it has no unfulfilled pre-conditions). Once there are no screens 
added in a certain step, the algorithm stops and discards all incomplete branches.

It should be pointed out that some of the tree structure is pre-computed to speed up the querying process at runtime. 
Any time a building block is inserted into the catalogue, the algorithm is executed following two approaches: 
\emph{forward search} and \emph{backward search}. The forward search approach finds the building blocks whose 
pre-conditions will be satisfied by the post-conditions of the new building block while the backward search 
finds the building blocks whose post-conditions will satisfy the pre-conditions of the recently
created building block.

\subsubsection{Results ranking}
\label{sssec:ranking}

This section explains the ranking techniques applied for the different discovery mechanisms.

The ranking algorithm for the simple approach applies the following rules:
\begin{inparaenum}[(1)]
	\item it gives a higher position to those building blocks which satisfy the highest number of pre-conditions;
	\item it prioritises building blocks created by the same user who is querying;
	\item it adjusts the rank by using the ratings given to the building blocks, and their popularity in terms of usage statistics;
	\item it weights the results according to non-functional features such as availability. This is only applied for what we call ``web service wrapper'', and it is calculated periodically by invoking the wrapped web services.
\end{inparaenum}

For the planning case, the objective is not only to produce a plan but also to satisfy user-specified preferences, or what is
known as \emph{preference-based planning}. The ranking algorithm:
\begin{inparaenum}[(1)]
 \item minimises the size of the plans, after removing the elements of the plan which are already in the canvas, so it gives
priority to the elements the user has already inserted,
 \item adjust the rank by using the rules 2, 3 and 4 used from the ranking algorithm of simple discovery based on 
pre- and post-conditions.
\end{inparaenum}
